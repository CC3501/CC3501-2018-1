% Template:     Template Controles LaTeX
% Documento:    Archivo principal
% Versión:      1.2.9 (05/02/2018)
% Codificación: UTF-8
%
% Autor: Pablo Pizarro R.
%        Facultad de Ciencias Físicas y Matemáticas
%        Universidad de Chile
%        pablo.pizarro@ing.uchile.cl, ppizarror.com
%
% Sitio web:    [http://latex.ppizarror.com/Template-Controles/]
% Licencia MIT: [https://opensource.org/licenses/MIT]

% CREACIÓN DEL DOCUMENTO
\documentclass[letterpaper,11pt]{article} % Articulo tamaño carta, 11pt
\usepackage[utf8]{inputenc} % Codificación UTF-8

% INFORMACIÓN DEL DOCUMENTO
\def\tituloevaluacion {Tarea 1}
\def\indicacionevaluacion {Fecha de entrega: 15/04/19 23:59 hrs}

\def\autordeldocumento {Pablo Pizarro}
\def\nombredelcurso {Modelación y Computación Gráfica para Ingenieros}
\def\codigodelcurso {CC3501-1}

\def\nombreuniversidad {Universidad de Chile}
\def\nombrefacultad {Facultad de Ciencias Físicas y Matemáticas}
\def\departamentouniversidad {Departamento de Ciencias de la Computación}
\def\imagendepartamento {departamentos/dcc}
\def\imagendepartamentoescala {0.2}
\def\localizacionuniversidad {Santiago, Chile}

% EQUIPO DOCENTE
\def\equipodocente {
	\textbf{Prof: Nancy Hitschfeld K.} \\
	Auxiliares: Pablo Pizarro R., Pablo Polanco Galleguillos, Mauricio Araneda H. \\
	Ayudantes: Iván Torres, María José Trujillo Berger \\
}

% CONFIGURACIONES
\input{lib/config}

% IMPORTACIÓN DE LIBRERÍAS
\input{lib/imports}

% IMPORTACIÓN DE FUNCIONES
\input{lib/function/control}
\input{lib/function/core}
\input{lib/function/elements}
\input{lib/function/equation}
\input{lib/function/image}
\input{lib/function/title}

% IMPORTACIÓN DE ESTILOS
\input{lib/styles}

% CONFIGURACIÓN INICIAL DEL DOCUMENTO
\input{lib/initconf}

% INICIO DE LAS PÁGINAS
\begin{document}

% CONFIGURACIÓN DE PÁGINA Y ENCABEZADOS
\input{lib/pageconf}

% ======================= INICIO DEL DOCUMENTO =======================

\sectionanum{Problema}

En una región del litoral central Chileno se planea construir una planta
industrial destinada a la refinación del petróleo. El diseño propuesto
considera una gran emisión de calor a la atmósfera, por lo cual en la
evaluación de impacto ambiental se le pide a usted, como alumno del curso de computación gráfica, modelar el comportamiento térmico de la atmósfera con la planta en operación. \\

\begin{wrapfigure}{r}{0.6\textwidth}
	\includegraphics[width=\textwidth]{perfil.pdf}
\end{wrapfigure}

\par Se le pide modelar un perfil del litoral (corte transversal en dirección
Este--Oeste) de 4 [km] de ancho y 2 [km] de alto parecido al de la figura. La
planta se ubicará en la playa (entendida como el borde entre el mar y las
montañas; en la figura se indica por una franja roja).

Por simplicidad consideraremos que la temperatura de la atmósfera $T(x,y)$ cumple la
ecuación de Poisson:
$$
\frac{\partial^2 T(x, y)}{\partial x^2} +
\frac{\partial^2 T(x, y)}{\partial y^2} = \rho(x,y)
$$
\\
En donde $\rho(x,y)$ es una \textbf{función cualquiera} que depende de la distancia $(x,y)$ con respecto al centro de la planta. Nótese que si $\rho(x,y)=0$ se tiene la ecuación de Laplace vista en clases. Esta función se debe proveer a los métodos que resuelvan el sistema como si fuese una variable (Ver código \ref{funporvalor} en Anexo).

\newpage
\sectionanum{Condiciones de borde}

La temperatura de la superficie del mar varía a lo largo del día de acuerdo a
la siguiente receta: T = 4\textordmasculine C entre las \texttt{00:00} y las
\texttt{08:00} hrs; luego T aumenta linealmente hasta alcanzar los
20\textordmasculine C a las \texttt{16:00}; y finalmente T decrece linealmente
hasta alcanzar T=4\textordmasculine C a las \texttt{24:00} (Ver figura \ref{temp} del anexo). Para el problema asumiremos que la temperatura superficial del mar varía en la forma descrita independiente de la temperatura atmosférica.

\newp La temperatura de la atmósfera (en ausencia de fuentes de calor que no sean la
superficie del mar) varía en el tiempo igual que la temperatura en la
superficie del mar pero además decae linealmente en 6\textordmasculine C cada
1000 [m]. Asumiremos que nuestra caja es suficientemente grande para que los
bordes de nuestro perfil no se vean afectados por la planta industrial.

\newp Por su parte, la temperatura del suelo en la región continental es constante e
igual a 20\textordmasculine C todo el día, excepto por sobre los 1800 [m],
donde hay nieve (que consideraremos a 0 \textordmasculine C).

\newp En cuanto a la planta, esta tiene chimeneas que cubren un ancho de 120 [m]
ubicada al nivel del mar. El comportamiento térmico de la chimenea a lo largo
del día esta descrito por la siguiente expresión (como función de la hora $t$):

$$
T = 450 \left(  \cos\left( \frac{\pi}{12}\cdot t \right) + 2 \right)
\quad [{\rm \degree C}];\ t \in [0,24]
$$

\sectionanum{Geografía}

El perfil geográfico a estudiar está detallado en la figura y contiene algunos
elementos aleatorios que dependen de \texttt{RRR} (los últimos 3 dígitos de su
\texttt{RUT}, antes del dígito verificador). En particular:

\begin{itemize}[label={--}]
	\item Partiendo del Oeste, la superficie del mar cubre $1200 + 400 \times
	0.\texttt{RRR}$ [m].
	\item A partir del borde costero, la planta industrial tiene chimeneas
	cubriendo un ancho plano de 120 [m] (la región roja de la figura).
	\item Luego de la planta hay una inclinación suave que aumenta 100 [m] de
	altura por cada 300 [m] que se recorre hacia el Este. Esta inclinación llega
	hasta 400 [m] a partir del borde costero.
	\item Luego de la pequeña inclinación, viene la cordillera de la costa
	que caracterizamos por dos cimas triangulares. La primera tiene una altura
	máxima de $1500 + 200 \times 0.\texttt{RRR}$ [m], la cual se ubica a 1200 [m]
	de la orilla del mar.
	\item A continuación hay un punto de menor altura: $1300 + 200
	\times\texttt{0.RRR}$ [m], el cual se ubica a 1500 [m] de la orilla del mar.
	\item Luego viene una segunda cima, más alta que la primera que alcanza
	$1850 + 100 \times 0.\texttt{RRR}$ [m] a 2000 [m] desde la orilla del mar.
	Recuerde que todos los puntos de la superficie que están a más de 1800 [m]
	están cubiertos de nieve a 0 \textordmasculine C.
	\item Decida Ud. qué hacer con el tramo que falta.
\end{itemize}

\sectionanum{Requerimientos}

\begin{enumerate}
	\item Elija una discretización apropiada para resolver el problema, identifique las condiciones de borde y las ecuaciones presentes en el sistema.
	\item Genere el terreno siguiendo las instrucciones previamente mencionadas en la sección \textit{Geografía}.
	\item Con $\rho(x,y)=0$, para las horas ${t=0, 8, 12, 16, 20}$ grafique la temperatura atmosférica (junto a las condiciones geográficas). Use el método de \textbf{sobre-relajación sucesiva} \footnote{De no hacerlo así obtendrá la nota mínima.} para resolver numéricamente el problema. Explore varios valores de $\omega$ entre 0 y 2, puede usar también la fórmula del $\omega$ óptimo vista en clases (Ver anexo), en donde $n$ y $m$ son el alto y ancho de la matriz.
	\item Con $\rho(x,y)=\frac{1}{\sqrt{x^2+y^2+120}}$, $x$ e $y$ en metros, resuelva el sistema para las horas anteriormente descritas.
\end{enumerate}

\sectionanum{Informe}

El informe debe incluir:

\begin{itemize}
	\item Breve descripción del problema. \textbf{(0.25pto)}
	\item Descripción de la generación del terreno. \textbf{(0.5pto)}
	\item Descripción de la solución numérica.  \textbf{(0.25pto)}
	\item Resultados obtenidos y posterior análisis de los mismos.  \textbf{(2.5pto)} \\Responda las siguientes preguntas:
	\vspace{-0.1cm}
	\begin{itemize}
		\item ¿Los resultados tienen sentido?
		\item ¿Afecta en algo el mar al sistema?
		\item ¿Las condiciones geográficas afectan al sistema?
		\item ¿En qué horas se obtiene la menor temperatura media en el sistema?
		\item ¿Cambia en algo la solución del sistema si $\rho(x,y) \neq 0$?
	\end{itemize}
	\item Use el método de sobre-relajación sucesiva con distintos $\omega$ y estudie el tiempo que tarda su programa en generar las soluciones tras un número de iteraciones \textit{fijo}, grafique $\omega$ vs tiempo de ejecución y comente los resultados.  \textbf{(2.0pto)}
	\item Conclusiones. \textbf{(0.5pto)}
\end{itemize}

\textbf{Formato PDF}: Los informes entregados en otro formato serán evaluados con nota \textbf{1.0} \\
\vspace{-0.2cm}
\par \textbf{Fecha de entrega}: Domingo 15 de Abril a las 23:59 hrs.

\newpage
\sectionanum{Condiciones de entrega}

\begin{itemize}
	\item La tarea es individual y las copias ser\'an penalizadas.
	\item Entregas solo vía U-Cursos.
	\item Adjuntar archivo \texttt{README.txt} con instrucciones de ejecución.
	\item Recuerde adjuntar \textbf{TODOS} los archivos en cada entrega, ya que los auxiliares y ayudantes sólo tienen acceso a la \'ultima entrega de su tarea.
	\item \textbf{NO} se aceptarán archivos atrasados una vez el plazo haya finalizado, aunque esto implique que su tarea no ejecute.
	\item Al hacer el informe usted debe decidir qué es interesante y
	agregar las figuras correspondientes. No olvide anotar los ejes, las unidades e incluir una \emph{caption} o título que describa el contenido de cada figura.
	
	\item No olvide indicar su RUT en el informe.
	
	\item Repartición de puntaje: 50\% implementación y resolución del problema; 50\% calidad del reporte entregado: demuestra comprensión del problema y su solución, claridad del lenguaje, calidad de las figuras utilizadas.
	\end {itemize}
	
\sectionanum{Recomendaciones}

\begin{itemize}
	\normalsize\item Revisar documentaci\'on y tutoriales de Python para obtener en detalle las funciones disponibles y ejemplos de uso.
	\normalsize\item Consultas a trav\'es del foro de U-Cursos o auxiliares.
	\normalsize\item Sea ordenado con su c\'odigo , agregue comentarios y utilice nombre \'utiles en sus variables y funciones.
	\normalsize\item Planifique su tiempo y comience su tarea con anticipaci\'on. No comience a programar directamente. Comprenda el problema, realice esquemas y plantee un algoritmo de soluci\'on.
	\normalsize\item Considere el uso de la programación orientada a objetos para el desarrollo de su solución (ver auxiliares).
\end{itemize}

\sectionanum{Anexos}

\insertimage[\label{temp}]{temperatura}{width=10cm}{Variación de temperatura del sistema}

\begin{sourcecode}[\label{funporvalor}]{Python}{Ejemplo paso función por argumento}
def rho(x, y):
	return 1 / (x**2 + y**2 + 120)**0.5

...

def iniciarIteracion(f):
	...
	matriz[i][j] = f(i/h, j/h)	# h dimension grilla

...

iniciarIteracion(rho) # Se pasa la funcion rho por variable
\end{sourcecode}

\insertequation{\omega = \frac{4}{2+\sqrt{4-\pow{\bigg(\cos{\big(\frac{\pi}{n-1}\big)}+\cos{\big(\frac{\pi}{m-1}\big)}\bigg)}{2}}}}

% FIN DEL DOCUMENTO
\end{document}