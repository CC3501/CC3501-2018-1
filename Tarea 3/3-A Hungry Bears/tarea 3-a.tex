% Template:     Template Controles LaTeX
% Documento:    Archivo principal
% Versión:      2.1.9 (14/06/2018)
% Codificación: UTF-8
%
% Autor: Pablo Pizarro R. @ppizarror
%        Facultad de Ciencias Físicas y Matemáticas
%        Universidad de Chile
%        pablo.pizarro@ing.uchile.cl, ppizarror.com
%
% Sitio web:    [http://latex.ppizarror.com/Template-Controles/]
% Licencia MIT: [https://opensource.org/licenses/MIT]

% CREACIÓN DEL DOCUMENTO
\documentclass[letterpaper,11pt]{article} % Articulo tamaño carta, 11pt
\usepackage[utf8]{inputenc} % Codificación UTF-8

% PLAZO DE ENTREGA
\def\plazoentrega {Domingo 19 de Agosto de 2018}

% INFORMACIÓN DEL DOCUMENTO
\def\tituloevaluacion {Tarea 3-A: Hungry Bears}
\def\indicacionevaluacion {Fecha de entrega: \plazoentrega} % Opcional

\def\autordeldocumento {CC3501-1}
\def\nombredelcurso {Modelación y Computación Gráfica para Ingenieros}
\def\codigodelcurso {CC3501-1}

\def\nombreuniversidad {Universidad de Chile}
\def\nombrefacultad {Facultad de Ciencias Físicas y Matemáticas}
\def\departamentouniversidad {Departamento de Ciencias de la Computación}
\def\imagendepartamento {dcc}
\def\imagendepartamentoescala {0.2}
\def\localizacionuniversidad {Santiago, Chile}

% EQUIPO DOCENTE
\def\equipodocente {
	\textbf{Prof: Nancy Hitschfeld K.} \\
	Auxiliares: Pablo Pizarro R., Pablo Polanco Galleguillos, Mauricio Araneda H. \\
	Ayudantes: Iván Torres, María José Trujillo Berger \\
}

% CONFIGURACIONES
\input{lib/config}

% IMPORTACIÓN DE LIBRERÍAS
\input{lib/env/imports}

% IMPORTACIÓN DE FUNCIONES
\input{lib/cmd/all}

% IMPORTACIÓN DE ESTILOS
\input{lib/style/all}

% CONFIGURACIÓN INICIAL DEL DOCUMENTO
\input{lib/cfg/init}

% INICIO DE LAS PÁGINAS
\begin{document}

% CONFIGURACIÓN DE PÁGINA Y ENCABEZADOS
\input{lib/cfg/page}

% ======================= INICIO DEL DOCUMENTO =======================

\sectionanum{Problema}

Los osos de Don Pedro se encuentran al otro lado del rio, alimentarlos no es tarea fácil, pues apenas tienen hambre, se enfurecen. Don Pedro ha inventado un simple pero eficaz sistema para enviarles comida sin atravesar el río. La idea es utilizar una honda gigante para lanzarles comida. \\

\insertimage{foto}{width=13cm}{*Imágenes referenciales, pues sus modelos deben ser tridimensionales.}

\textbf{En esta tarea, debe utilizar OpenGL para dibujar. Como lenguaje base se recomienda utilizar Python.}

\sectionanum{Modelos (4 puntos)}

Dibuje 4 osos distintos utilizando OpenGL y distintas estrategias:

\begin{itemize}
	\item Uno construido en base a figuras GLUT, utilizando al menos unas 4.
	\item Uno generado a partir de un fichero STL, OBJ u otro formato (Puede encontrar figuras en internet).
	\item Un tercer oso dibujado por usted en base a especificación manual de vértices y polígonos.
	\item Para el cuarto oso puede repetir cualquiera de los procedimientos anteriores o utilizar una mezcla de ellos.
\end{itemize}

Utilizando las estrategias que usted estime conveniente, dibuje 6 alimentos y una honda que permita lanzarlos.

\newp Dibuje un escenario utilizando la(s) estrategia(s) que usted estime conveniente. Deben distinguirse 2 plataformas, un río entre ellas y algún fondo. Formule su diseño para una visualización 3D.

\newp Configure una fuente de luz. Todos sus modelos deben utilizar sombreado, ya sea FLAT o SMOOTH.

\newp Implemente un programa que permita visualizar sus modelos de comida, sus osos y su honda. De forma análoga, implemente otro programa que permita visualizar el escenario y su animación.

\sectionanum{El juego (2 puntos)}

Para implementar correctamente el juego debe considerar lo siguiente:

\begin{itemize}
	\item Inicialmente hay un alimento en la honda, y 4 osos al otro lado del rio, en distintas posiciones.
	\item En cada turno, el jugador podrá lanzar un alimento, si es capturado por el oso, el jugador gana 10 puntos y el oso se retira de la zona. Si el turno anterior también se alimento un oso, el jugador gana 5 puntos adicionales.
	\item Al final de cada turno debe imprimir en la consola la puntuación actual del jugador.
	\item El jugador solo dispone de 6 alimentos, si no alcanza a alimentar a todos los osos, pierde la partida.
	\item El juego puede ser pensado en cualquiera de los siguientes modos:
	\begin{itemize}
		\item Una visualización 3D en corte transversal: En este caso, el lanzamiento del alimento debe proceder como: Con las flechas es posible estirar el elástico de la honda, lo que le dará mayor intensidad al disparo. Con las flechas es posible rotar la dirección que apunta la honda. Al presionar \quotes{espacio}, el alimento es lanzado.
		
		\item Una visualización 3D donde la cámara se encuentre tras la onda, y el alimento pueda ser direccionado libremente. Utilizando las flechas del teclado, fijar la dirección del disparo rotando hacia los lados y hacia arriba y abajo. Con las teclas \quotes{a} y \quotes{z} es posible controlar la intensidad del disparo (estiramiento del elástico). Al presionar ‘espacio’, el alimento es lanzado.
	\end{itemize}

	\item Al lanzar el alimento, este debe respetar las leyes de la física, es decir, debe tener movimiento parabólico.
	\item No se preocupe de la colisión del alimento con el suelo, espere a que el alimento salga del escenario y considere dicho caso como fallido.
	\item Si el oso atrapa el alimento, ambos desaparecen del escenario. Si no lo atrapa, el oso sigue en juego pero desaparece el alimento.
\end{itemize}

\sectionanum{Bonus Track (máximo 1 punto sobre la nota)}

Elija entre los siguientes:

\begin{itemize}
	\item Día y noche \textbf{(1 punto)}: configure dos fuentes de luz, que representarán el sol y la luna (dibújelos utilizando esferas GLUT u otra estrategia). Ambas deben moverse en el cielo alternando entre el día y la noche. El cielo debe cambiar de color en este proceso. Procure que toda la secuencia ocurra en menos de 40 segundos.
	\item Cámara que sigue al alimento \textbf{(0,5 puntos)}: implemente una cámara que se ubique tras el alimento (lo debe seguir durante el lanzamiento). Esta cámara se debe visualizar mientras se mantenga presionado \quotes{m}.
	\item Alimento múltiple \textbf{(0,5 punto)}: Implemente un alimento que se separe en varios otros cuando el usuario presione la tecla \quotes{x}.
\end{itemize}

\sectionanum{Presentación-Informe}

Se espera que en el informe de esta tarea se explique la metodología llevada a cabo para resolver el problema, la estructura de su código, los resultados y una breve discusión sobre los resultados obtenidos. 

\newp En particular especifique los modelos utilizados tanto en el lanzamiento (cálculo de la velocidad inicial), como mientras el alimento se encuentre en el aire.

\subsectionanum{Observaciones}

\begin{itemize}
	\item \textbf{Plazo de Entrega: \plazoentrega \ hasta las 23:59. No se aceptan atrasos.}
	\item No se reciben tareas por otro medio que no sea la sección habilitada en u-cursos.
	\item Entregue TODOS los archivos que sean necesarios para la correcta ejecución de su programa.
	\item Puede reenviar su tarea dentro del plazo. En este caso, procure subir nuevamente TODOS sus archivos. El equipo docente sólo tiene acceso a la última entrega que usted realiza.
	\item La presentación-informe DEBE estar en formato .pdf, no se revisarán otros formatos.
	\item Las tareas se revisarán utilizando Python 2.7 y sus librerías (instaladores en ucursos).
	\item Para consultas, se ofrece soporte en el lenguaje Python. También puede utilizar C++ o Java (junto a OpenGL), pero va por su cuenta y debe presentar el código entregado funcionando en su computador.
\end{itemize}

\subsectionanum{Hints}

\begin{itemize}
	\item Planifique su tiempo y comience su tarea con anticipación. No comience directamente a programar. Identifique fragmentos de algoritmos, formule clases y objetos útiles, si lo necesita, diseñe máquinas de estados.
	\item Construya su programa de manera incremental, esto es, vaya respaldado su código de forma que siempre disponga de algo que \quotes{funcione}.
	\item Se recomienda utilizar pequeños programas que simplemente dibujen cada una de las partes del juego (objetos, personajes, escenario, etc..). Del mismo modo, puede construir pequeños programas que vayan añadiendo las interacciones entre los distintos actores.
	\item Una buena metodología es diseñar una clase por cada tipo de actor que interfiera en su programa.
\end{itemize}

% FIN DEL DOCUMENTO
\end{document}